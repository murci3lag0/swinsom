One of the goals of machine learning is to eliminate tedious and arduous repetitive work. The manual and semi-automatic classification of millions of hours of solar wind data from multiple missions can be replaced by automatic algorithms that can discover, in mountains of multi-dimensional data, the real differences in the solar wind properties. In this paper we present how unsupervised clustering techniques can be used to segregate different types of solar wind. We propose the use of advanced data reduction methods to pre-process the data, and we introduce the use of Self-Organizing Maps to visualize and interpret 14 years of ACE data. Finally, we show how these techniques can potentially be used to uncover hidden information, and how they compare with previous manual and automatic categorizations.