\subsubsection{Model Amaya-21}

\paragraph{General overview}

Fig.\ref{fig:maps} (A) shows the distribution of code words in the latent space. The feature and components maps in the bottom row show in addition that lattice nodes share common attributes with their neighbors. The regularity in the colors of the feature map confirms that the SOM keeps its most important feature: organization. This is clear by the proximity of neighboring code words in the latent space, marked with red lines. We expect then to find common patterns in all the following maps in this section.

In the same figure the panels (B) and (C), and the component maps in the bottom row, show black and white lines indicating the relative distances between the lattice nodes: thicker lines represent larger inter-node distances. This shows that there are sections of the map (groups of code words in the latent space) that can form separate groups. This group separation has been highlighted in the `Feature map': the $k$-means clustering of the code words divides the space following the inter-nodal divisions. The classified points in the 3D latent space are shown in the third row of Fig.\ref{fig:clustering}.

The `Hit map' in Fig.\ref{fig:maps} presents a good distribution of points among all the lattice nodes, except in regions isolated from the rest. These zones represents solar wind types that have atypical properties. One of the goals of the DSOM is to cover those isolated zones where rare events can be classified in separate nodes. In contrast, the classic SOM method tends to cluster the code words in regions of high density, troubling the categorization of rare events, like ICMEs, ejecta or magnetic clouds, in the solar wind data.

Fig.\ref{fig:datarange} shows the distribution of the SOM training data in the normalized range $\left[0..1\right]$. This is a violin plot superposed by a box plot, showing the data distribution for each one of the features listed in table \ref{tab:features}. Normalization of the data is performed using the maximum and minimum values of each feature, but outliers (extremely large or small values) can hinder the use of particular features. In any classification problem, outlier detection and elimination is extremely important. The figure shows that all our data points are well represented in the data. Even in some cases, like for feature 13 (Alfv\'en Mach number), where the distribution has a small width, it still covers a significant part of the total range. This figure also shows that feature 6 (cross-helicity, $\sigma_c$) has a bimodal distribution, with two peaks close to the limits. A significant part of the data lays close to the $\sigma_c = \pm 1$ limit. The cross-helicity is a measure of the Alfv\'enicity of the solar wind, representing the direction and intensity of the propagation of Alfv\'enic fluctuations. At the Earth orbit this an indication of the origins of the solar wind in the north or south hemispheres of the Sun.

Feature 20 (Pearson auto-correlation of the magnetic field magnitude) also shows a large distribution function, with a marked peak near one. This quantity was calculated for a window of time of six hours and a time lag of one hour. It shows the extent to which the values of the magnetic field have changed in one hour. High autocorrelation values represent situations in which the magnetic field does not change during the window, i.e. values in the window at time $t$ are the same as values shifted by one hour, $t-1$. Completely uncorrelated signals, produced by random changes in time, will produce autocorrelation values close to 0 (0.5 after normalization), i.e. the data in the window at time $t$ is different from the data in the shifted window. Positive (negative) values represent a signal with a periodicity of one hour. Additional time lags could be used to create extra features, but here we use only one to test its effectiveness.

The features selected for this model have not followed a meticulous vetting process. We included features inspired from previous publications and new interesting additions. Our goal in this work was to test if the data transformation into an encoded latent space can account for redundant or un-interesting features. This is a very useful property for data sets where expert knowledge is not available. It also shows how the SOM can point to features that don't have added value. As we will see in the next sections, the method converges to meaningful classes, even when some of the features used turn out to be not very relevant. We will perform in a future publication a more detailed selection of the features, based on the experience of human experts. 

\paragraph{Class and Feature Maps}

Lattice nodes are characterized by their weights (code). Applying a reverse transformation, followed by a re-scaling, we obtain their values in the original N-dimensional space. Fig.\ref{fig:compmap} shows the DSOM clustering and the 21 solar wind properties associated to each lattice node. We have clustered the nodes in eight classes. This is a subjective selection inspired by other works in the literature. In our case the clustering leads to contiguous groups of nodes.

%%% ME
\textcolor{blue}{We can try to attribute a physical significance to the classes by analysing, together, the characteristic  features of each class and the solar wind identified as part of each class in Figure \ref{fig:timeseries}.\\
Class 5 can be mapped to transient events, CMEs and ejecta. It presents very high values of the oxygen ion charge state (`O7to6'), a feature that \citet{Zhao2009, Stakhiv2015, Xu2015b} associate to CME plasma. It is also characterised by high solar wind velocity, cross-helicity $\sigma_c \sim 0$, high values of magnetic field magnitude and Alfv\'{e}n speed. These features are usually associated to explosive transient activity~\citep{Roberts2020, Xu2015b}. Figure~\ref{fig:timeseries}, panel c shows that class 5 indeed maps well to the Richardson and Cane, UNH and CfA CME catalogs.\\
We can then use the cross-helicity, $\sigma_c$ feature, to identify two groups of classes: the ones with mostly positive (1 and 3) and mostly negative cross-helicity (0, 2, 4, 6, 7). As already done in ~\citet{Roberts2020}, we associate them to solar wind plasma originating from areas with different magnetic polarity, respectively northern and southern sector. Inspection of Figure~\ref{fig:timeseries} confirms this association. \\
Among the classes with negative $\sigma_c$, class 7 can be quite confidently associated with coronal hole plasma. It is characterised by the very low values of the O7+/ O6+  ratio that  \citet{Zhao2009, Stakhiv2015, Xu2015b} associate with plasma originating from open magnetic field lines. It also exhibits high wind speed, low proton density and proton density variability, high absolute values of cross-helicity and equipartition levels of the residual energy $\sigma_r$ (telltale signs of Alfv\'{e}nicity), high proton entropy and moderately high values of Alfv\'{e}n speed, high proton temperature and high proton temperature variability. These are characteristics widely associated to fast coronal holes solar wind plasma. Inspection of Figure~\ref{fig:timeseries} again supports this identification.\\
Class 4 and 0 are both possibly composed of a mix of slow Alfv\'{e}nic wind \citep{DAmicis2015} and ``conventional" slow and intermediate speed wind. Slow Alfv\'{e}nic wind shares coronal hole origin and a number of characteristics related to the origin (O7+/O6+ ratio, low density values, high $|\sigma_c|$ and low $\sigma_r$ values) with the fast Alfv\'{e}nic wind. The main difference, apart, of course, from the speed, is the proton temperature, which tends to be lower in Alfv\'{e}nic and ``conventional" slow wind with respect to the Alfv\'{e}nic fast wind. 
The parameters that are usually used to distinguish between slow and fast wind (speed, density, proton entropy, proton temperature) span a quite large interval in both classes 0 and 4, and in fact point to the presence of a mix of slow and fast wind in both. The main difference between class 0 and 4 is given by the very high values of the residual energy $\sigma_r$ in class 4, which point to kinetically dominated structures.\\
Class 6 is characterised by features generally associated with intermediate and slow wind of streamer belt origin: intermediate values of O7+/O6+ ratio, intermediate and slow speed, high proton density, low absolute value of cross-helicity, low values of magnetic field magnitude, proton entropy, Alfv\'{e}n speed, proton temperature. \\
Class 2 is characterised primarily by very high values of the Alfv\'{e}nic Mach number. It is a class with a very low number of hits (see the hit map), and rarely spotted in Figure ~\ref{fig:timeseries}.\\
Among the classes associated with positive cross-helicity, class 1 and 3, we associate class 3 to coronal hole origin and class 1 to streamer belt origin. Our class 3, orange, maps quite closely to the ``red" class in Figure 1 and 2 of \citet{Roberts2020}, associated there to coronal hole plasma from sectors of positive polarity. Class 3 indeed shows high absolute values of cross-helicity and near-zero values of residual energy, as expected form Alfv\'{e}nic wind from coronal holes. We notice, however, that its O7+/ O6+ ratio, velocity, density, proton entropy and proton temperature values are somehow less �coronal hole-like� that the ones observed in class 7 for the opposite polarity. A qualitative difference between the wind from sectors of opposite magnetic polarity can be seen in Figure ~\ref{fig:timeseries}, and is already remarked upon in \citet{Roberts2020}. While we are quite surprised that it extended to the large time interval ($\sim$ 4 solar cycles) that we used to generate the SOM, we reserve to conduct further analysis on the topic in the future. }
%%% original
%The maps show the properties that differentiate each of the eight classes. The most remarkable class is number 5. This class clearly show high values of oxygen ion charge state (`O7to6') and compositional ratio (Fe/O). It presents a high values of solar wind speed, average iron charge, large magnetic field magnitudes, temperature ratios, and strong changes in all the range quantities.  All of this is indicative of transient events (ICMEs, ejecta).

%With one of the lowest values of cross-helicity, class 0 in the center of the class map has a magnetic field almost aligned with the ecliptic plane, high proton temperatures and low average ion charge. The values of $\sigma_c$ close to -1, suggesting the possibility of an IMF connected to the northern hemisphere of the Sun, where the magnetic polarity was predominantly negative during cycle 23. This configuration points towards an origin north of the streamer belt.

%On the contrary, class 1 presents a positive $\sigma_c$, pointing towards a southern hemisphere origin, and a low $\sigma_r$ indicating that the flow fluctuating energy is mainly carried by the velocity. A very low value of the `Delta' parameter, $\theta_{B,RTN}$, also points towards a southward oriented IMF. This might indicate a slow geoeffective solar wind.

%Solar wind of SOM class number 2 carries its fluctuating energy equally between the velocity and the magnetic field ($\sigma_r$ around 0). But it also presents the highest Alfv\'en Mach number. This wind has simultaneously slow speeds, low temperatures and small magnetic field magnitude, which is rarely observed, as shown by the hit map in Fig.\ref{fig:maps}. This slow cold wind can have origins in the streamer belt.

%The slow solar wind from class 3 presents a low average iron charge and carries similar dynamic and magnetic energies. However, it is characterized by a very high value of the cross-helicity, indicating a possible origin in the southern hemisphere of the Sun. Classes 1 and 3 share may similarities, with the only glaring difference in the value of their magnetic field autocorrelation. This is also evident from the observation of the high variability in the density and the magnetic field during the previous 6 hours, as shown by the range maps.

%A low density, low cross-helicity wind is associated to class 4. This class presents the highest residual energies of all the classes, i.e. the fluctuating energy of this wind is carried predominantly by the flow. This can indicate an origin on the norther hemisphere, above the streamer belt.

%The slow and cold wind of class 6 is very similar to the wind observed in class 2, but with a higher proton density and a slightly negative `Delta'. Class 6, as it was the case with class 2, can be part of the slow geoeffective solar wind.

%Class 7 shows very low ion charge ratios and average ion charges. However it presents high temperature and temperature jumps, without a density jump, pointing towards a classification as a magnetic cloud. This is also implied by the maps of magnetic field range and by the very low autocorrelation.

The identification of all these properties have been done by simple visual examination of the SOMs. 14 years of data are compressed in a single extremely powerful and useful figure. Here we not only classify data points, but we can also explain why they were classified together. This is the essence of eXplainable and interpretable Artificial Intelligence (XAI) which we consider indispensable in any machine learning study. 

\paragraph{Solar Wind Fingerprints}

SOMs show the variability of solar wind and how it can be visually characterized. The SOM is a helpful guide in the study of the different types of solar wind, but is not necessarily an objective, unbiased and final classification method. SOMs open the possibility for a fast visual characterization of large and complex data sets.

The short analysis of the different SOM classes, performed in the previous section, was informed by the data presented in Fig.\ref{fig:classesdatarange}. Each row corresponds to a single solar wind class, represented by its 21 features using box plots. The colors correspond to the class number (0 to 7 from top to bottom). The first column has been built from a classification using $k$-means, the second with GMM and the third with DSOM. Here we can find again the properties described in the previous section. We call these plots class `fingerprints'.

Different classification methods lead to different classes with different fingerprints. A visual inspection of the fingerprints is much more difficult to interpret than the SOMs. \citep{Roberts2020} and \citep{Xu2015b} performed detailed descriptions of particular solar wind classes based on the mean values observed in each subset of points. But Fig.\ref{fig:classesdatarange} shows that some features can have a very large distribution. For example the values of the solar wind speed, feature 2, have a very large spread on all the classes and all classifications, except for class 7 in the GMM classification. Other features with large fingerprint spread includes the cross-helicity (6) and the residual energy (7). These are produced by the bimodal nature of the features, as shown in the violin plots of Fig.\ref{fig:datarange}.

In our analysis of the solar wind classification in the previous section, we took into account the level of dispersion of each feature in the class fingerprint. It is noticeable that in the $k$-means classification multiple classes have very similar fingerprints, at the exception of a single feature. For example classes 6 and 7 have similar characteristics except for features 6 (cross-helicity). The same is true for classes 0 and 2 in the GMM classification, where the largest difference is the spread of feature 20. SOM classes tend to present fingerprints that are more variable. This is a consequence of the use of the dynamic version of the SOM that does not agglomerate nodes in zones with high density of points. On the contrary, $k$-means and GMM will tend to put more points in high density zones, creating very similar class fingerprints.

\paragraph{Time series}
A more classical analysis of the solar wind can be performed by experts using visual inspection of the properties of the solar wind during time windows. Fig.\ref{fig:timeseries} presents a window of time of four months, from the beginning of May 2003 to the end of September 2003. We have plotted entries in the Richardson and Cane, UNH and CfA catalogs on top of time series of the solar wind speed in panels (A), (B) and (C). These plots have been colored by the class number of the $k$-means, GMM and SOM classes.

The polarity of the solar wind can be observed in panel (D). Changes from red to green are associated with crossings of the heliospheric current sheet. The z-component of the magnetic field is plotted in blue (positive) and red (negative) in panel (E).

Panel (F) shows the evolution of the $O^{7+}/O^{6+}$ ratio using a dotted black line in logarithmic scale. The Zhao classification boundaries (see table \ref{tab:swtypes}) are plotted using a black continuous line. The red area corresponds to `non-coronal hole origin' solar wind, and data points receive this classification if the dotted line enters the red zone. If it stays above it, the solar wind is considered an ICME. If the curve drifts bellow the red zone, the wind is considered to have origins in a coronal-hole.

The three time series in panels (A), (B) and (C) show that multiple techniques can be used for the clustering of solar wind properties. But SOMs allow fast visualization and interpretation not available in other clustering methods. All time series have a strong tendency to group the solar wind in two groups, depending on the heliospheric sector. This is due to the importance of the cross-helicity in the data set and its bimodal distribution.

We see how the characterization of the SOM classes performed in the previous sections is now expressed in the time series plot. Our analysis suggests that class 5 of the SOM correspond to ICMEs. In the time series in panel (C) clear correlations with the catalog entries for from \citep{Richardson2012} are visible. Panel (D) marks regions of sector reversal bined also in class 5. Fig.\ref{fig:tsfeatures-som} shows that this class is mainly associated with very strong oxygen ion ratios, high iron average charge and strong magnetic fields, associated with ejecta.

The time series also shows that SOM class 5 contains data with near zero cross-helicity, above average iron to oxygen ratios and a high oxygen ion ratio. This was also found by \citep{Roberts2020} and is believed to be correlated with flux ropes connected t both ends to the Sun, with waves propagating in both directions along the field lines. This is also visible in the maps of Fig.\ref{fig:compmap}. The hit map also indicates the low number of hits in the corresponding lattice nodes.

Class 4 was expected to be solar wind with north solar hemisphere origins. This is confirmed by the plots in Fig.\ref{fig:tsfeatures-som}, where the cross-helicity is close to -1. This figure also shows that classes 0 and 4 present similar profiles in this window of time. Their fingerprints only differ by the intensity of the magnetic field, the range of the proton density and the spread of the magnetic field autocorrelation distribution. Class 4 presents in these time series plots higher values of $\sigma_r$, pointing towards a more Alfv\'enic solar wind than the wind in class 0.

Class 2 presents very low numbers in the hit map. In the time series, it is only visible in small zones before or after shock entries in the UHN and CfA catalogs, around the events of end-May and beginning and end of August. This entries are more visible in Alfv\'en velocity plot of Fig.\ref{fig:tsfeatures-som}, suggesting higher Mach number values.

Classes 6, 7 and 2 are very difficult to analyze in these plots: while class 7 seems to apear in multiple random locations in the graphs, classes 6 and 2 show only in very few locations. It is not difficult to understand that a manual selection of the different types of solar wind, even by an expert, is an extremely complex task. The tools we present in this paper will support future expert analysis, in particular on the identification of rare and unique events.

\paragraph{Maps of Empirical Classifications}
SOM allows visual analysis of previously published results. In this section we show how the Xu and Zhao classifications activate different nodes of the SOM. We use two properties of the SOM simultaneously: the size of the lattice nodes will represent the number of hits for a particular class, and the color will represent one property of the solar wind.

To perform this analysis, instead of using the full data set, we extract 3 subsets corresponding to the entries categorized as CHW, ICME and NCHW in the Zaho catalog. Each one of these three subsets is passed through the Amaya-21 model and we observe how each one activates the SOMs. All properties are normalized between zero and one, using the maximum and minimum values for each feature in the full data set, so we can perform comparisons among all the subsets.

Fig.\ref{fig:SWtypeZ} shows the SOMs of the three Zhao classes produced by the Amaya-21 model. The maps can be interpreted in the following way: CHW, ICME and NCHW classes have different number of hits. These solar wind types activate different nodes in the lattice. Each row in the figure would show a different activation of the map for each one of the three subsets, however class CHW and NCHW do not seem clearly activate different nodes in this model. The color for each class is almost the same inside each map but should be different between the classes. Here the differences between CHW and NCHW are still not very clear. The values of oxygen ion state ratio and the solar wind speed do not seem to play an important role in the automatic classification of our model. Our goal is to work, in combination with solar experts, on the generation of a more sophisticated model that can accurately reproduce the Zhao classes in our maps.

However, it is clear that the ICME class is mainly contained in the zone corresponding to class 5, which has been previously identified as such in the map and time series analysis sections above. Here the total number of hits is only 445, which explains why it is so difficult to observe in the time series. This is an additional benefit of using SOMs: we are able to detect important data points that can easily be overlooked with other methods.

In a similar way, Fig.\ref{fig:SWtXu} shows the SOMs of the Xu classification. This time we used four subsets of the data set, each one corresponding to a different Xu class. Once again `ejecta' is confined to the region of SOM class 5, and `sector reversal origin' solar wind activates some of the nodes corresponding to the non-coronal hole wind in the Zhao classification. This same zone is also overlapping with `streamer belt origin' zones in the Xu classification. This class seems to be included in the SOM class 2, and in part in class 3. It is possible to isolate singular nodes and study more in detail all their characteristics, but this is out of the scope of the present work and will be presented in a future pubication.

The separation of the Xu classes is also not perfect in this model. We have tested other models in which the separation is more clear. Those models were based on different number of features and time ranges. An example of such model is presented in the next section.

\subsubsection{Model Roberts-8}
The same techniques used for model Amaya-21 were applied to this model. The only difference between these two is the amount of data used and the selected initial features. We can see that these two modifications can have an important impact in the final results. Fig.\ref{fig:modelR} (A) shows the volume integrated point density and the distribution of the code words. This is a projection in the latent space after transformation using an autoencoder. The `hit map' and the `feature map' shows a clear segregation of points, allowing for a proper splitting of the data set.

The time series in Fig.\ref{fig:modelR} shows that the model can differentiate zones of high and low speed, as well as zones of polarity inversion and some of the shocks and ICMEs. The use of less features gives more dominance to properties with larger distribution spread, like the cross-helicity. It is clear from the time series that class 4 and 5 are dominant northwards and southwards of the HCS respectively. Class 3 in this model is associated with transient events, including ICMEs and sector reversals.

For simplicity we will not present a detailed analysis of the Roberts-8 model. We would like only to point an important difference with model Amaya-21: Fig.\ref{fig:SWtXuRoberts} shows how the solar wind classifications by Xu and Zhao are interpreted by the model. First, it is important to notice that all classes activate nodes predominantly in different SOM classes. Second, the small variations in colors inside each map demonstrates that the classes are well represented by the main properties proposed by Xu and Zhao. One exception is the class `ejecta' that shows uneven values of proton specific entropy, $S_p$, and temperature ratio $T_{\text{exp}}/T_p$.

coronal hole solar wind classes by Xu and Zhao activate exactly the same nodes in the map, corresponding to class 4 in the Roberts-8 model. ICME from Zhao is considered as a subset of the ejecta class from Xu, while the NCHW class from Zhao contains the sector reversal class from Xu.

A careful selection of features and the data range of the models can produce a particularly powerful tool for the analysis of solar wind information. We are currently working towards the creation of an accurate solar wind classification system based on the developments presented here.